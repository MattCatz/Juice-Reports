\section{Phase 2}
The first phase of development ultimately aims to produce a simple, working prototype. The second phase instead aims to add more features to the prototype, or at least to noticeably improve the initial design. Assuming development in the first phase is successful, there are multiple directions for development that would significantly improve our system. The two main directions are the creation and integration of a first response system and improved detection capabilities using sensor fusion.

The design for a first response system can be made arbitrarily complex, and the speed of development later in the semester will determine the exact kind of system. An example of a simple system would be that of an automatically deployed raft. Using information from the sensors, the unit (or another unit that is activated) could place itself underneath the target and inflate itself. This would provide immediate and reliable safety, all while being among the simplest designs. More sophisticated designs could utilize networks of units to perform a rescue task in unison, as well as utilize other rescue mechanisms. For example, a first response system could be designed to coordinate a kind of mechanical "lift" that can attach itself to the target and lift it to safety (say the edge of the pool). While these kinds of designs are much more complicated and cross-disciplinary in nature, the simpler design will be attempted first. If time permits, various elements of the more advanced design (like the lifting mechanism) could be integrated with the design.

The other main direction of design objectives involves the integration of multiple sensors to greatly enhance the detection capabilities of the system. Some sensors ubiquitous to industry like lidar and sonar would provide localization assistance and target identification and tracking. Sonar systems have existed for a relatively long time, namely in the use of underwater navigation and intelligence systems in submarines. Lidar has been used extensively in recent times, namely in autonomous vehicles and ADAS (Advanced Driver-Assistance Systems) packages. The fact that these two sensors have had numerous, effective uses in tasks similar to those of our design show the potential for them to improve upon our design. Other simpler sensors exist and could be used to the same effect, such as optical and thermal cameras. The team has a relatively large amount of experience in signal and image processing, so the main candidates for sensor fusion are sonar and optical cameras.

While lidar, sonar, and cameras would improve the detection and tracking capabilities of the system, other types of sensors could make the system more useful in general. The addition of chemical and ecologically-motivated sensors could make the system marketable as more than a safety product. From inside the pool (or perhaps other types of bodies of water), the system could provide information about the weather, water health, and various information about the surroundings. With the rise of the IoT, products like this are invaluable not only to the consumer, but scientific and research agencies greatly benefit from the existence and development of such products. Because these kinds of sensors are decidedly independent of the tracking and detection technology, they could easily be integrated into the system, as their integration would not interfere with its primary functions.