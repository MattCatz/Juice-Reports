\section{Product Testing}

There are multiple factors to consider throughout the testing phase. Because it is unknown which factors will contribute to \juicy’s performance, they must all be carefully monitored until they are ruled out. During the initial testing phase, the testing environment will be as controlled as possible in order to simplify matters. Once \juicy is performing perfectly at this simplified level, one new obstacle at a time will be introduced in order to guarantee desired results for all possible conditions.

Initial testing will be conducted in an indoor pool in order to exclude interference from inclement weather. For the first several tests, \juicy will be placed in the same location in the pool, and the simulated victim will be dropped into the pool at a nearby, unchanging location. After determining that \juicy is operating well under these standard conditions, the location of the victim’s immersion will be moved further away in small increments until \juicy’s maximum range has been determined. In the next phase of testing, the shape, depth, and temperature of the pool will all be varied. Finally, outdoor elements such as wind and rain will be simulated to ensure that \juicy still functions properly.

When conducting similar tests, the Consumer Product Safety Commission (CPSC) discovered that one to two year old children are at greatest risk of drowning in a home swimming pool. The weight of the average one year old child is eighteen pounds, which is the weight used by the CPSC in their product safety tests. However, the average child becomes mobile around the age of six months, which would decrease the absolute minimum testing weight to approximately sixteen pounds. In the future, \juicy may be refined to detect disturbances caused by even smaller weights, such as that of a small dog.